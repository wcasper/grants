
%%%%%%%%% MASTER -- compiles the 6 sections

\documentclass[11pt,letterpaper]{article}

%%%%%%%%%%%%%%%%%%%%%%%%%%%%%%%%%%%%%%%%%%%%%%%%%%%%%%%%%%%%%%%%%%%%%%%%%
\pagestyle{plain}                                                      %%
%%%%%%%%%% EXACT 1in MARGINS %%%%%%%                                   %%
\setlength{\textwidth}{6.5in}     %%                                   %%
\setlength{\oddsidemargin}{0in}   %% (It is recommended that you       %%
\setlength{\evensidemargin}{0in}  %%  not change these parameters,     %%
\setlength{\textheight}{8.5in}    %%  at the risk of having your       %%
\setlength{\topmargin}{0in}       %%  proposal dismissed on the basis  %%
\setlength{\headheight}{0in}      %%  of incorrect formatting!!!)      %%
\setlength{\headsep}{0in}         %%                                   %%
\setlength{\footskip}{.5in}       %%                                   %%
%%%%%%%%%%%%%%%%%%%%%%%%%%%%%%%%%%%%                                   %%
\newcommand{\required}[1]{\section*{\hfil #1\hfil}}                    %%
\renewcommand{\refname}{\hfil References Cited\hfil}                   %%
\bibliographystyle{plain}                                              %%
%%%%%%%%%%%%%%%%%%%%%%%%%%%%%%%%%%%%%%%%%%%%%%%%%%%%%%%%%%%%%%%%%%%%%%%%%

%PUT YOUR MACROS HERE

\usepackage{amsmath}
\usepackage{amssymb}
\usepackage{amsthm}
\usepackage[margin=1.0in]{geometry}
\usepackage{enumerate}

\theoremstyle{definition}
\newtheorem{goal}{Project}
\newtheorem{subgoal}{Subgoal}[goal]
\newtheorem{thm}{Theorem}
\newtheorem{quest}{Question}

\newcommand{\bbr}{\mathbb{R}}
\newcommand{\bbc}{\mathbb{C}}
\newcommand{\vocab}[1]{\textbf{#1}}
\newcommand{\Gr}{\text{Gr}}

\pagestyle{empty}
%\includeonly{NSFsumm}

\begin{document}

\required{Project Summary}
\begin{center}
William Riley Casper
\end{center}

\begin{enumerate}[(1)]
\item{\textbf{Overview.}}
This proposal expands the study of noncommutative algebras of differential operators arising naturally from (1) prolate-spheroidal integral operators, (2) commuting fractional differential operators, and (3) matrix-valued orthogonal polynomials.
Via noncommutative algebra and algebraic geometry, the following projects address important gaps in the literature on bispectrality, orthogonal matrix polynomials, and integrable systems.
\begin{goal}\label{goal1}
Perform asymptotic spectral analysis of self-adjoint integral operators with the prolate spheroidal property, parameterized by symmetric points in a Calogero-Moser space.
Leverage algebraic geometry to obtain a universal description of the spectra of the family of integral operators defined by points in a Calogero-Moser flow.
\end{goal}
\begin{goal}\label{goal2}
Extend Burchnall-Chaundy Theory and Krichever correspondence to commuting algebras of fractional differential operators.
Leverage this to parameterize solutions of KP hierarchy in terms of the action of a Picard group on sections of the dual of a jet bundle.
\end{goal}
\begin{goal}\label{goal3}
Construct explicit bispectral Darboux transformations for orthogonal matrix-valued polynomials of Hermite type.
Obtain a precise description of the corresponding adelic-type classifying space.
Leverage this description to obtain examples of matrix-valued prolate-spheroidal integral operators.
\end{goal}
\item{\textbf{Intellectual Merit.}}
All three projects are united by Sato's grassmannian $\Gr$, the adelic grassmannian $\Gr^{ad}$, and their generalizations:
the union of Calogero-Moser spaces is homeomorphic to the adelic grassmannian so that Calogero-Moser flows are KP flows; bispectral Darboux transformations are classified in known cases by generalizations of $\Gr^{ad}$; geometric points of $\Gr$ parameterize vector bundles on projective curves.
Commuting pairs of fractional differential operators correspond to points of $\Gr$ which are not geometric, but which via the PI's extended Krichever correspondence are given a geometric interpretation in terms of sections of the dual of the jet bundle $J^\infty(\pi)$ on line bundle $\pi: E\rightarrow X$ over an algebraic curve.

Motivated by the natural formulation of KP flows of geometric points in terms of the action of the Picard group of the associated projective curve, the PI proposes to determine a similar formulation for commuting fractional differential operators via a natural action of the Picard group on jet bundles.  This paves the way for new exact formulations of solutions to equations in the KP hierarchy in terms of algebro-geometric data in Project 2.
Additionally our projects are linked by the presence of Fourier algebras, noncommutative operator algebras attached to points in a (generalized) adelic grassmannian of bispectral functions.
This algebra is key to the construction of the integral operators in Project 1, whose spectral data will lend new insight into eigenvalue distributions from random matrix ensembles.
Furthermore the Fourier algebra was the main ingredient in the classification of the noncommutative bispectral Darboux transformations Project 3, and will be essential in their proposed adelic parameterization.
%Attaining the above goals will greatly contribute to our current understanding of three important topics: orthogonal matrix polynomials, bispectral operators, and higher dimensional McKay correspondence.
%The PI's results on the first goal will refine the PIs recent classification theorem for the matrix Bochner problem and will directly link the behavior of $\tau$-spherical functions on compact rank $1$ Gelfand pairs $(G,K)$ to orthogonal matrix polynomials.
%This platform by which both phenomena may be studied will play a pivotal role at the intersection of integrable systems and representation theory.

%In pursuit of the second goal, the PI will solve a long-standing open problem in the literature on bispectral differential operators.
%Additionally, the method used will expand on very recent applications of Fourier-Mukai transforms in the study of commuting differential operators.
%
%The third goal provides a novel perspective from which to study derived equivalence for quotient spaces.
%Specifically, we will develop certain generalizations of Cherednik algebras which will act as noncommutative crepant resolutions of symplectic quotient singularites.
%Using this method we can hope to address an important conjecture of Reid on derived eqivalence for McKay correspondence in higher dimensions.

\item{\textbf{Broader Impacts.}}
The PI has demonstrated his commitment ot having a positive broader impact on his community through his volunteer work teaching mathematics to underprivileged groups, mentorship of younger students, and service to the greater mathematics community.
Specific examples include his lecturing in the Freedom Education Project and his mentoring of students in summer schools and undergraduate research groups.
He has previously and will continue to broadly disseminate his research in the form of journal publications and frequent talks at domestic and international conferences.
In the future the PI will work to increase his impact on the mathematical community by actively searching for opportunities to help organize conferences and create and grow research seminars.
\end{enumerate}
\end{document}

