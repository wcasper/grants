
%%%%%%%%% MASTER -- compiles the 6 sections

\documentclass[11pt,letterpaper]{article}

%%%%%%%%%%%%%%%%%%%%%%%%%%%%%%%%%%%%%%%%%%%%%%%%%%%%%%%%%%%%%%%%%%%%%%%%%
\pagestyle{plain}                                                      %%
%%%%%%%%%% EXACT 1in MARGINS %%%%%%%                                   %%
\setlength{\textwidth}{6.5in}     %%                                   %%
\setlength{\oddsidemargin}{0in}   %% (It is recommended that you       %%
\setlength{\evensidemargin}{0in}  %%  not change these parameters,     %%
\setlength{\textheight}{8.5in}    %%  at the risk of having your       %%
\setlength{\topmargin}{0in}       %%  proposal dismissed on the basis  %%
\setlength{\headheight}{0in}      %%  of incorrect formatting!!!)      %%
\setlength{\headsep}{0in}         %%                                   %%
\setlength{\footskip}{.5in}       %%                                   %%
%%%%%%%%%%%%%%%%%%%%%%%%%%%%%%%%%%%%                                   %%
\newcommand{\required}[1]{\section*{\hfil #1\hfil}}                    %%
\renewcommand{\refname}{\hfil References Cited\hfil}                   %%
\bibliographystyle{plain}                                              %%
%%%%%%%%%%%%%%%%%%%%%%%%%%%%%%%%%%%%%%%%%%%%%%%%%%%%%%%%%%%%%%%%%%%%%%%%%

%PUT YOUR MACROS HERE

\usepackage{amsmath}
\usepackage{amssymb}
\usepackage{amsthm}
\usepackage[margin=1.0in]{geometry}
\usepackage{enumerate}

\theoremstyle{definition}
\newtheorem{goal}{Project}
\newtheorem{subgoal}{Subgoal}[goal]
\newtheorem{thm}{Theorem}
\newtheorem{quest}{Question}

\newcommand{\bbr}{\mathbb{R}}
\newcommand{\bbc}{\mathbb{C}}
\newcommand{\vocab}[1]{\textbf{#1}}
\newcommand{\Gr}{\text{Gr}}

\pagestyle{empty}
%\includeonly{NSFsumm}

\begin{document}

\required{Project Summary}
\begin{center}
William Riley Casper
\end{center}

\begin{enumerate}[]
\item{\textbf{Overview.}}
The NSF DMS annually accepts proposals for research projects in algebra and number theory, with funding for research related equipment, travel expenses, conference support and summer salary for up to three years (for 2020, see NSF PD 20-1264).
The following research summary is intended for pursuing funding for research through this program during the 2021 funding cycle.
The project below fits with the NSF's goal of generating new knowledge, and with the specific program's target of supporting research in algebra, algebraic and arithmetic geometry, number theory, and representation theory.

\item{\textbf{Summary.}}
In the 1830's, the scientist John Scott Russell described racing on horseback to follow strange, fast-moving water waves generated by boats in a shallow channel.
Today, almost 200 years later, Russell's waves remain a topic of great interest.
They are examples of solitons: nonlinear waves described by the Kadomtsev-Petviashvili (KP) equation.
Incredibly, the KP equation arises in the otherwise completely unrelated topic of algebraic geometry via Shiota's famous solution of the Schottky problem.
In fact, this fascinating connection between the KP equation an algebraic geometry allows us obtain explicit solutions of the KP equation, or more generally the KP hierarchy, in terms of certain special functions in algebraic geometry called theta functions.

Solutions of the KP hierarchy in general are classified by the motion of points in an infinite dimensional space, Sato's grassmannian $\Gr$, under a natural group action.
Motivated by these connections, the proposed research project will discover new connections with the KP hierarchy, expanding on recent results of the PI which fit the same theme (see Publications Related to the Proposed Project in the Brief CV).
Additionally, our project aims to incorporate related undergraduate research project opportunities.

To motivate our work and highlight the potential for undergraduate collaboration, consider classic boardgame Battleship, but with ``radar", where you scan the enemy fleet, giving you the sum of positions occupied by enemy ships in a specified row or column.
Assuming that you've scanned your opponents fleet so that you know all the row and column sums, how do you now reconstruct the fleet?
Abstractly, this problem is really about studing the linear transformation taking a battleship fleet to the collection of row and column sums.

The linear transformation described in the previous paragraph is an example of a (discrete) integral operator with a special property called the ``prolate spheroidal property", that it commutes with a (discrete) differential operator.
In recent work, the PI and his collaborators established the existence of huge class of integral operators with the ``prolate spheroidal property", vastly expanding the handful of examples previously known coming from spherical functions and random matrix theory.
These integral operators are constructed by using certain special functions called bispectral functions, whose classification is again described by $\Gr$, and thus linked with the KP equation.
Our proposed project will leverage this connection to impose \emph{dynamics} on prolate-spheroidal operators for the first time, with important implications to both random matrix theory and integrable systems.

\end{enumerate}
\end{document}

