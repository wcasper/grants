
%%%%%%%%% MASTER -- compiles the 6 sections

\documentclass[12pt,letterpaper]{article}

%%%%%%%%%%%%%%%%%%%%%%%%%%%%%%%%%%%%%%%%%%%%%%%%%%%%%%%%%%%%%%%%%%%%%%%%%
\pagestyle{plain}                                                      %%
%%%%%%%%%% EXACT 1in MARGINS %%%%%%%                                   %%
\setlength{\textwidth}{6.5in}     %%                                   %%
\setlength{\oddsidemargin}{0in}   %% (It is recommended that you       %%
\setlength{\evensidemargin}{0in}  %%  not change these parameters,     %%
\setlength{\textheight}{8.5in}    %%  at the risk of having your       %%
\setlength{\topmargin}{0in}       %%  proposal dismissed on the basis  %%
\setlength{\headheight}{0in}      %%  of incorrect formatting!!!)      %%
\setlength{\headsep}{0in}         %%                                   %%
\setlength{\footskip}{.5in}       %%                                   %%
%%%%%%%%%%%%%%%%%%%%%%%%%%%%%%%%%%%%                                   %%
\newcommand{\required}[1]{\section*{\hfil #1\hfil}}                    %%
\renewcommand{\refname}{\hfil References Cited\hfil}                   %%
\bibliographystyle{plain}                                              %%
%%%%%%%%%%%%%%%%%%%%%%%%%%%%%%%%%%%%%%%%%%%%%%%%%%%%%%%%%%%%%%%%%%%%%%%%%

%PUT YOUR MACROS HERE

\usepackage{amsmath}
\usepackage{amssymb}
\usepackage{amsthm}
\usepackage[margin=1.0in]{geometry}
\usepackage{enumerate}
\usepackage{mathptmx}

\theoremstyle{definition}
\newtheorem{goal}{Project}
\newtheorem{subgoal}{Subgoal}[goal]
\newtheorem{thm}{Theorem}
\newtheorem{quest}{Question}

\newcommand{\bbr}{\mathbb{R}}
\newcommand{\bbc}{\mathbb{C}}
\newcommand{\vocab}[1]{\textbf{#1}}
\newcommand{\Gr}{\text{Gr}}

\pagestyle{empty}
%\includeonly{NSFsumm}

\begin{document}

\subsection*{\hfil 2. Expected Outcomes and Methods of Dissemination (322 words)\hfil}

The PI expects the proposed research project to result in 
\begin{enumerate}[(a)]
\item a paper submitted to a Q1 research journal in pure mathematics, coauthored with an undergraduate research student
\item presentations at research conferences, including the MAA Undergraduate Research Poster Session in January 2022
\item research talks in one or more departmental seminars at CSUF, such as the problem solving seminar
\end{enumerate}
Additionally, the strong results obtained from this research will bolster the PI's application for external funding via a grant in algebra and number theory from the NSF (PD 20-1264), due October 2021.  In particular, the close collaboration with undergraduates will simultaneously enhance the PI's NSF grant application in the area of Broader Impacts.  At the same time, this work is consistent with the University's Strategic Plan 2018-2023 to ``provide a transformative educational experience and environment for all students" by prividing high-impact co-curricular experiences in the form of real research experience with publishable results.  In the future, the PI will leverage this collaborative opportunity to establish a solid track record of undergraduate research essential for applying for external funding for additional undergraduate research in the future.  In this way, the current project is also consistent with the university's goal to increase on-campus student employment, internships, and professional development opportunities.

Starting in Fall 2021, the PI in collaboration with other faculty from the math department will design and implement a series of research and professional development seminars aimed at junior and senior math students planning to apply for graduate school, summer research internships (eg. REUs), or positions in industry.  Specifically, the seminars will include both a mix of talks by students and faculty on recent or ongoing research projects, as well as practical workshops on creating professional resumes and writing strong application essays.  In the future, the PI intends to apply for external funding to establish a formal summer mathematics research program combining both student/faculty research collaboration and extended professional development opportunities.


\end{document}

